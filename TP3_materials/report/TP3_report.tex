\documentclass[a4paper]{article} 
\input{style/head.tex}

%-------------------------------
%	TITLE VARIABLES (identify your work!)
%-------------------------------

\newcommand{\yourname}{Balthazar Neveu | Jamy Lafenetre}
\newcommand{\youremail}{balthazarneveu@gmail.com | jamy.lafenetre@ens-paris-saclay.fr}
\newcommand{\assignmentnumber}{3}

\begin{document}

%-------------------------------
%	TITLE SECTION (do not modify unless you really need to)
%-------------------------------
\input{style/header.tex}


%-------------------------------
%	ASSIGNMENT CONTENT (add your responses)
%-------------------------------


\section*{Question 1}
\begin{figure}[ht]
  \centering
  \includegraphics[width=0.3\linewidth]{figures/cc_normals_10cm.png}
  \includegraphics[width=0.3\linewidth]{figures/cc_normals_50cm.png}
  \includegraphics[width=0.3\linewidth]{figures/cc_normals_2m.png}
  \includegraphics[width=0.3\linewidth]{figures/cc_normals_10cm_stopsign.png}
  \includegraphics[width=0.3\linewidth]{figures/cc_normals_50cm_stopsign.png}
  \includegraphics[width=0.3\linewidth]{figures/cc_normals_2m_stopsign.png}
  \caption{From left to right: 10cm, 50cm, 2m radiuses used to compute normals}
  \label{fig:cc_normals}
\end{figure}
If the neighborhood radius is too small, we get noisy normals.
If the neighborhood radius is too large, we get smoothed normals (makes edges curvy).
Stop sign normals are not recovered correctly (considered like a horizontal plane like the ground around it) when radius is too big.


\section*{Question 2}
Picking the right radius is a tradeoff between noisy normals and smoothing.


\section*{Question 3}
\begin{figure}[ht]
  \centering
  \includegraphics[width=.9\linewidth]{figures/cc_normals_PCA_50cm_bigger_points.png}
  \caption{50cm, radiuses local PCA used to compute normals}
  \label{fig:local_pca}
\end{figure}


\section*{Question 4}
\begin{figure}[ht]
  \centering
  \includegraphics[width=.9\linewidth]{figures/Schema.png}
  \caption{Lidar acquisition leads to a specific points distribution. when combined with a fixed amount
  of nearest neighbors used to compute normals, this leads to visible artficats in the normal maps.
  } 
  \label{fig:lidar}
\end{figure}

The combination of lidar acquisition and the use of a fixed amount of nearest neighbors 
to compute normals leads to an anisotropic distribution of samples among each query.
Fitting a plane on such a set is not suited, a plane cannot be properly fitted with nearest neighbors.

The effect is also midly apparent when using a fixed radius because the radius may integrate other "beams" of the lidar.


\begin{figure}[ht]
  \centering
  \includegraphics[width=0.46\linewidth]{figures/cc_normals_PCA_r=50cm_smooth.png}
  \includegraphics[width=0.46\linewidth]{figures/cc_normals_PCA_50cm_k=30_nearest.png}
  \includegraphics[width=0.46\linewidth]{figures/cc_normals_PCA_r=50cm_v2.png}
  \includegraphics[width=0.46\linewidth]{figures/cc_normals_PCA_50cm_k=30_v2.png}
  \caption{Left: 50cm radius local PCA used to compute normals. 
  Right: 30 nearest neighbors used to compute normals.} 
  \label{fig:local_PCA_neighbor}
\end{figure}




\begin{figure}[ht]
  \centering
  \includegraphics[width=0.46\linewidth]{figures/zoom_r=0_1.png}
  \includegraphics[width=0.46\linewidth]{figures/zoom_k=6.png}
  \caption{Left: 10cm radius local PCA used to compute normals. 
  Right: 6 nearest neighbors used to compute normals.} 
  \label{fig:local_PCA_neighbor_zoom}
\end{figure}
Issue when the radius is too small, not enough candidates to estimate the plane.

\end{document}

