\documentclass[a4paper]{article} 
\input{style/head.tex}

%-------------------------------
%	TITLE VARIABLES (identify your work!)
%-------------------------------

\newcommand{\yourname}{Balthazar Neveu | Jamy Lafenetre}
\newcommand{\youremail}{balthazarneveu@gmail.com | jamy.lafenetre@ens-paris-saclay.fr}
\newcommand{\assignmentnumber}{3}

\begin{document}

%-------------------------------
%	TITLE SECTION (do not modify unless you really need to)
%-------------------------------
\input{style/header.tex}


%-------------------------------
%	ASSIGNMENT CONTENT (add your responses)
%-------------------------------


\section*{Question 1}
\begin{figure}[ht]
  \centering
  \includegraphics[width=0.3\linewidth]{figures/cc_normals_10cm.png}
  \includegraphics[width=0.3\linewidth]{figures/cc_normals_50cm.png}
  \includegraphics[width=0.3\linewidth]{figures/cc_normals_2m.png}
  \includegraphics[width=0.3\linewidth]{figures/cc_normals_10cm_stopsign.png}
  \includegraphics[width=0.3\linewidth]{figures/cc_normals_50cm_stopsign.png}
  \includegraphics[width=0.3\linewidth]{figures/cc_normals_2m_stopsign.png}
  \caption{From left to right: 10cm, 50cm, 2m radiuses used to compute normals}
  \label{fig:cc_normals}
\end{figure}
If the neighborhood radius is too large, the observed geometry is no longer local and the normals are therefore oversmoothed.
Figure \ref{fig:cc_normals} highlights this phenomenon: the normals of the car and of the stop sign pole cannot be recovered accurately when the radius is too large.
If the neighborhood radius is too small, the estimated normal are inacurate because of the noise, but also because of the local variance separating samples.
Independantly of the measurment noise, samples are indeed scattered randomly and separated by a certain distance; estimations based on a narrow neighborhood can therefore
be prone to aliasing, which can appear as structured artefacts in the normal estimations. As highlighted by Robert Cook's "Stochastic Sampling in Computer Graphics 
", aliasing in such a stochastic sampling setting can also translate as a white noise in the normal estimation. This is visible in Fig \ref{fig:cc_normals} where both structured pattern
and noise appear in the normal whe the radius is too short.

% If the neighborhood radius is too small, we get noisy normals.
% If the neighborhood radius is too large, we get smoothed normals (makes edges curvy).
% Stop sign normals are not recovered correctly (considered like a horizontal plane like the ground around it) when radius is too big.


\section*{Question 2}
The optimal radius must satisfy a tradeoff between noisy normals and smoothing.


\section*{Question 3}
\begin{figure}[ht]
  \centering
  \includegraphics[width=.9\linewidth]{figures/cc_normals_PCA_50cm_bigger_points.png}
  \caption{50cm, radiuses local PCA used to compute normals}
  \label{fig:local_pca}
\end{figure}


\section*{Question 4}
% \begin{figure}[ht]
%  \centering
%  \includegraphics[width=.9\linewidth]{figures/Schema.png}
%  \caption{Lidar acquisition leads to a specific points distribution. when combined with a fixed amount
% of nearest neighbors used to compute normals, this leads to visible artficats in the normal maps.
%  } 
%  \label{fig:lidar}
% \end{figure}

We compare our results for a radius of 50cm or for $k=30$ neighbours in Figure \ref{fig:local_PCA_neighbor}. The $k-nn$ methods 
produced more accurate results, and is able to recover the step of the sidewalk as well as the building corner. We posit that this is due
to the localy-variable density of points: samples can be more sparse in flat areas than on textures. Therefore, $k-nn$ acts like
an adaptive radius, and can naturally integrate a narrower area around details, where samples are abundant). The non uniformity of the
sample density can also be explained by the sampling process of the sensor.

In Figure \ref{fig:local_PCA_neighbor_zoom}, we reduce the radius to 10cm and the number of neighbours to 6. In this setting, $k-nn$ is less
aliased because it is able integrate a sufficient amount of points, contrary to the radius method that cannot properly fit a plane.


% The combination of lidar acquisition and the use of a fixed amount of nearest neighbors 
% to compute normals leads to an anisotropic distribution of samples among each query.
% Fitting a plane on such a set is not suited, a plane cannot be properly fitted with nearest neighbors.

% The effect is also midly apparent when using a fixed radius because the radius may integrate other "beams" of the lidar.
% Issue when the radius is too small, not enough candidates to estimate the plane.

\begin{figure}[ht]
  \centering
  \includegraphics[width=0.46\linewidth]{figures/cc_normals_PCA_r=50cm_smooth.png}
  \includegraphics[width=0.46\linewidth]{figures/cc_normals_PCA_50cm_k=30_nearest.png}
  \includegraphics[width=0.46\linewidth]{figures/cc_normals_PCA_r=50cm_v2.png}
  \includegraphics[width=0.46\linewidth]{figures/cc_normals_PCA_50cm_k=30_v2.png}
  \caption{Left: 50cm radius local PCA used to compute normals. 
  Right: 30 nearest neighbors used to compute normals.} 
  \label{fig:local_PCA_neighbor}
\end{figure}




\begin{figure}[ht]
  \centering
  \includegraphics[width=0.46\linewidth]{figures/zoom_r=0_1.png}
  \includegraphics[width=0.46\linewidth]{figures/zoom_k=6.png}
  \caption{Left: 10cm radius local PCA used to compute normals. 
  Right: 6 nearest neighbors used to compute normals.} 
  \label{fig:local_PCA_neighbor_zoom}
\end{figure}

\section*{Question Bonus}
We represent on Figure \ref{fig:verticality} the verticality which describes, as suggested by the name, the verticality of the predicted normals.
On Figure \ref{fig:descriptors}, we simultaneously represent the linearity, the planarity and the sphericity. The linearity is maximal when a single
eigenvalue numericaly dominates the 2 others, therefore indicating the presence of an elongated 1D vectorial space (also commonly called line).
The planarity is maximal when 2 eigenvalues are equivalently scaled, while numerically dominating the third, indicating a plane. The sphericity is maximal when 
all eigenvalues are numerically comparable, indicating that there is no favorised axis. This is typically the case for corners, textured areas, and objects of very small size.
Figure \ref{fig:descriptors} highlighs how these descriptors do indeed describe the aforementioned features.



\begin{figure}[ht]
  \centering
  \includegraphics[width=0.46\linewidth]{figures/verticality.png}
  \caption{"verticality" descriptor computed for a radius of 20cm. Dark shades indicate a verticality of 0 and white shades a verticality of 1.
  The descriptor accurately indicates surfaces orthogonal to the vertical axis.} 
  \label{fig:verticality}
\end{figure}

\begin{figure}[ht]
  \centering
  \includegraphics[width=0.46\linewidth]{figures/lin_plan_sphe.png.png}
  \caption{Local image descriptors computed for a radius of 20cm. The linearity (represented in red) accurately describes elongated objects such as poles.
  The planarity (in green) accurately represents flat areas. The sphericity (in blue) represents corners, textures and small objects such as a tiled roof, the base of pole and car mirrors.
  Note that the 3 descriptors never overlap, by design.} 
  \label{fig:descriptors}
\end{figure}






\end{document}

