\documentclass[a4paper]{article} 
\input{style/head.tex}

%-------------------------------
%	TITLE VARIABLES (identify your work!)
%-------------------------------

\newcommand{\yourname}{Balthazar Neveu | Jamy Lafenetre}
\newcommand{\youremail}{balthazarneveu@gmail.com | jamy.lafenetre@ens-paris-saclay.fr}
\newcommand{\assignmentnumber}{2}

\begin{document}

%-------------------------------
%	TITLE SECTION (do not modify unless you really need to)
%-------------------------------
\input{style/header.tex}

%-------------------------------
%	ASSIGNMENT CONTENT (add your responses)
%-------------------------------


\section{Question 1: ICP}



\section{Question 2: Rigid transformation estimation}
\begin{itemize}
  \item 2 sets of matched points, $\text{RMS} = 0.$.
Works perfectly because the input points are perfectly matched (no noise) and we use a closed form solution for the rigid transformation estimation.
  \item ICP struggles because of a bad initialization and gets stuck in a local minimum (translation is not too bad).
\end{itemize}

For the real scene "Notre Dame des champs", we do not even know if the points are matched (we don't have the same amount of points.) 

% \begin{figure}[ht]
%     \centering
%     \includegraphics[width=0.5\linewidth]{figures/temperature_sampling.png}
%     \caption{Random sampling on the distribution using the temperature parameter. Temperature = 0 is equivalent to the argmax sampling.}
%     \label{fig:random_sampling}
% \end{figure}



\end{document}