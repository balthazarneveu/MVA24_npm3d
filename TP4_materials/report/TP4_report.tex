\documentclass[a4paper]{article} 
\addtolength{\hoffset}{-2.25cm}
\addtolength{\textwidth}{4.5cm}
\addtolength{\voffset}{-3.25cm}
\addtolength{\textheight}{5cm}
\setlength{\parskip}{0pt}
\setlength{\parindent}{0in}

\usepackage[square,sort,comma,numbers]{natbib}
\usepackage{blindtext} % Package to generate dummy text
\usepackage{charter} % Use the Charter font
\usepackage[utf8]{inputenc} % Use UTF-8 encoding
\usepackage{microtype} % Slightly tweak font spacing for aesthetics
\usepackage{amsthm, amsmath, amssymb} % Mathematical typesetting
\usepackage{float} % Improved interface for floating objects
\usepackage{hyperref} % For hyperlinks in the PDF
\usepackage{graphicx, multicol} % Enhanced support for graphics
\usepackage{xcolor} % Driver-independent color extensions
\usepackage{pseudocode} % Environment for specifying algorithms in a natural way
\usepackage[mmddyy]{datetime} % Uses YEAR-MONTH-DAY format for dates

\usepackage{fancyhdr} % Headers and footers
\pagestyle{fancy} % All pages have headers and footers
\fancyhead{}\renewcommand{\headrulewidth}{0pt} % Blank out the default header
\fancyfoot[L]{} % Custom footer text
\fancyfoot[C]{} % Custom footer text
\fancyfoot[R]{\thepage} % Custom footer text
\newcommand{\note}[1]{\marginpar{\scriptsize \textcolor{red}{#1}}} % Enables comments in red on margin

\DeclareMathOperator*{\argmin}{arg\,min}

%----------------------------------------------------------------------------------------


%-------------------------------
%	TITLE VARIABLES (identify your work!)
%-------------------------------

\newcommand{\yourname}{Balthazar Neveu | Jamy Lafenetre}
\newcommand{\youremail}{balthazarneveu@gmail.com | jamy.lafenetre@ens-paris-saclay.fr}
\newcommand{\assignmentnumber}{4}

\begin{document}

%-------------------------------
%	TITLE SECTION (do not modify unless you really need to)
%-------------------------------
\fancyhead[C]{}
\hrule \medskip
\begin{minipage}{0.295\textwidth} 
\raggedright
\footnotesize
\yourname \hfill\\
\youremail
\end{minipage}
\begin{minipage}{0.4\textwidth} 
\centering 
\large 
Lab session \# \assignmentnumber\\ 
\normalsize 
NPM 2024\\ 
\end{minipage}
\begin{minipage}{0.295\textwidth} 
\raggedleft
\today\hfill\\
\end{minipage}
\medskip\hrule 
\bigskip




%-------------------------------
%	ASSIGNMENT CONTENT (add your responses)
%-------------------------------


\section*{Question 1}
The octree depth is the parameter that has the most impact on the result, but also on the
computation time. Bigger values lead to a deeper tree and more scale for the SPR solving.
It can also increase drastically the number of voxels present on the finest scale, which can
lead the linear system solving to become non-tractable.

If the number of scale is too low (we show the result for n=6), the voxel grid is too
coarse to represent the geometry and we can therefore notice blocky artifacts.
If the number of scale is too high (n=10) then the grid may become too fine for the sample
density. The implicit discretization of integrals done in the Galerkine solver no longer
holds, as they are too few samples within each voxel. This results in high frequency artefacts.

We can also tweak the number of samples per node to ensure that each nodes has enough samples.
The built-in plugin claims that higher valus suchs as 15 are preferable for noisy clouds.
This is indeed plausible, as the low frequency Bezier kernel used in the solver will integrate
many samples withing a single voxel to attenuate noise.
For clouds without noise, using such high values will prevent the voxel hierarchy to
populate the finest level, therefore resulting in some local blocky artefacts.
Additional artefacts can be visible around structures featuring very high frequencies,
because the leaf voxels are too coarse for the local geometry.

The regularisation weight (dubbed "point weight" in cloud compare) drive the screening
impoprtance of the SPR. Lower values give smooth surface, whereas very high values will
"overfit" the cloud and the noise. 


The given point cloud has very little noise, and the bunny is a very smooth closed structure.
With a maximum tree depth of 8, a maximum sample-per-node value of 1.5 and a point wight value
of 2, we were able to obtain the most satisfactory reconstruction. In our case a satisfactory
reconstruction is a mesh that preserves the global smoothness os the bunny, without losing
the few sharper details and without display grid artefacts or spiky artefacts.

The resulting mesh is composed 373 800 triangles.

\section*{Question 2}

\section*{Question 3}
We compare our handcrafted Hoppe implementation with the SPR plugin of CloudCompare.
Qualitative results can be seen on figure # TODO, and the quantitative results can 
are reported in table \ref{table:HoppeTable}. Our implementation takes 22 seconds to run
on the bunny cloud, which is mostly due to the single threaded kd-tree implementation we use.
If we neglect the interaction with the tree and do the processing a GPU,
our reconstruction takes around 17ms. A better optimisation could realistically lower this
overhead to a few milliseconds.
SPR takes between 1 and 2 seconds to compute a surface.

\begin{table}
    \centering
    \begin{tabular}{|c|c|c|}
      \hline
         & SPR & Hoppe \\
      \hline
      Computation time & ~ 1s & ~10ms \\
      Number of triangles & 32 168 & 373 800 \\
      Quality of reconstruction & High & Poor \\
      \hline
    \end{tabular}
    \caption{Comparison between our Hoppe implementation and SPR}
    \label{tab:HoppeTable}
  \end{table}

\section*{Question 4}
We run the same comparison as before with our handcrafted implementation of IMLS,
and compare the result on figure # TODO and in table \ref{table:IMLS}. We use
the same strategy as before to approximate the execution time. The result is
smoother and more satisfactory that with Hoppes, at the cost of a few
more operations. We can observe that IMLS is able to recover smooth structures such
as the ears of the bunny accurately, contrary to Hoppes. However, only SPR can recover small details
such as the nose.

\begin{table}
    \centering
    \begin{tabular}{|c|c|c|c|}
      \hline
         & SPR & Hoppe & IMLS\\
      \hline
      Computation time & ~ 1s & ~ 10ms & ~ 100ms\\
      Number of triangles & 373 800 & 32 168 & 77 164\\
      Quality of reconstruction & High & Poor & Medium\\
      \hline
    \end{tabular}
    \caption{Comparison between our implementation of Hoppes, IMLS and SPR}
    \label{tab:IMLSTable}
  \end{table}


\end{document}

