\documentclass[a4paper]{article} 
\input{style/head.tex}

%-------------------------------
%	TITLE VARIABLES (identify your work!)
%-------------------------------

\newcommand{\yourname}{Balthazar Neveu | Jamy Lafenetre}
\newcommand{\youremail}{balthazarneveu@gmail.com | jamy.lafenetre@ens-paris-saclay.fr}
\newcommand{\assignmentnumber}{6}

\begin{document}

%-------------------------------
%	TITLE SECTION (do not modify unless you really need to)
%-------------------------------
\input{style/header.tex}


%-------------------------------
%	ASSIGNMENT CONTENT (add your responses)
%-------------------------------


\section*{Question 1}

We train our model on ModelNet40. It has 1.708M parameters. Using a learning rate of $1e-2$, we
are only able to reach an accuracy of $12\%$ after 75 epochs (convergence is achieved).

The accuracy of a random classifier would be $2.5\%$, so this result is not dramatic.
However it cannot be considered a good classifier.

\section*{Question 2}
On ModelNet40, we reach an accuracy of $86\%$ after 75 epochs using a learning rate of $1e-3$.
These results are considerably higher than using a simpl MLP. A Pointnet without Tnets can be used
like a satisfactory classifier.

\section*{Question 3}
On ModelNet40, we reach an accuracy of $84\%$ after 100 epochs using a learning rate of $5e-3$.
These results are slightly under what could be obtained without Tnet.

\section*{Question 4}
We propose two sources of data augmentation. First, we apply three random scaling factors centered around 1 in the x, y and z direction. 
Second, we remove a random proportion of the points (around $10\%$) and replace them by duplicating other points.

Using the same parameters as before, the impact of this data augmentation is not meaningful and seems to slightly hinder the accuracy.

\end{document}

